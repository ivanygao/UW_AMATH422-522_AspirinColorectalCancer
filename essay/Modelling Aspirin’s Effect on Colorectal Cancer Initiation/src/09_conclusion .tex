\section{Conclusion}

Our goal was to identify the crypt transition with the largest relative impact on cancer incidence, given the mathematical model presented in the Wang et. al paper. We chose to approach this problem by first modeling crypt populations as a stage-structured system represented by a stage-structured matrix, and then computing the sensitivities to compare the impact of different crypt transitions on the total crypt population. There are two important caveats to this method.

First, the transition matrix requires constant values in each cell in order to compute numerical sensitivity values. To fulfill this constraint, the rate of crypt transitions must be linear. Since the model presented in the paper includes a nonlinear crypt competition parameter, we needed to find a way to eliminate or approximate this term through linearization. Biologically, this crypt competition parameter represents the competition for nutrients and space that real crypts encounter on the colonic surface, which prevents unlimited growth via fission. By calculating the Jacobian of the system around the 0 vector equilibrium point, we ended up with a linear system that was very similar to the original nonlinear system, except the crypt competition term is gone. Since we computed the sensitivity analysis on this linear approximation of the system and not the original system, another question of interest emerged from this work: can this system be faithfully approximated through the 0 equilibrium Jacobian linearization? We attempt to explore this question by plotting and comparing the cancer incidence curves generated by running each model. In this plot, we observe that the linear approximation adopts a clear sigmoidal curve within the natural human lifespan (80 years), where it predicts a 100\% probability of cancer by age 60. On the other hand, the growth of the original nonlinear model is stunted by the crypt competition parameter, allowing the model to predict cancer incidence that more accurately reflects collected epidemiological data. 

Second, the sensitivity analysis does not directly tell us anything about the crypt transitions’ impact on cancer incidence, but rather their impact on the total crypt population. Total crypt population can be represented as $\sum_{i=1}^{6} n_i$, while cancer incidence (I) is represented as $\dot{I} \propto (R_{36}*n_3 + R_{56}*n_5)*(1-P)$, where P is the probability of cancer incidence.

Due to the fission behavior of stage 3 and stage 5 crypts that is defined by the linear model, crypt population is generally correlated with the rate of acquiring a cancerous crypt. However, to definitively compare the sensitivities of each crypt transition on cancer incidence, we needed to empirically test it in a model that directly calculates cancer incidence. We attempted to do this by plotting and comparing the cancer incidence curves described by the mathematical model, created in response to perturbing each crypt transition rate by $-10\%$. This procedure was done in the linear model, where we found that both the empirical testing and sensitivity analysis pointed to the same conclusion: the $n_3 \rightarrow n_3$ crypt transition is the most important player, aka how much each $n_3$ crypt specifically is persisting from year to year in the crypt population. 

To find out whether our sensitivity analysis can be applied to the original nonlinear model, we performed the same procedure in the nonlinear model. The result from this testing conflicted with the sensitivity analysis. It suggests that the general behavior of end-stage precancerous crypts, $n_3$ and $n_5$, exert the most impact on overall cancer incidence, as opposed to the single $n_3 \rightarrow n_3$ transition suggested by the sensitivity analysis. This discrepancy points to the secondary question proposed earlier in this section: can this system be faithfully approximated through the 0 equilibrium Jacobian linearization? According to this analysis, no. Since the crypt transitions in the linear approximation have demonstrably different impacts on cancer incidence compared to the original model, it suggests that the linear approximation is modelling a different system entirely that is not similar to the original model. At least in the context of plotting cancer incidence over a natural human lifetime, this linear model $\not\approx$ the nonlinear model.


%Overall, the results imply that the linear system doesn't accurately capture the internal function or behavior of the original nonlinear model presented in the Wang et al paper.