\section{Introduction}

Colorectal cancer (CRC) causes the second most cancer deaths in the United States (USA). This year in the USA, it is expected that more than 150,000 people will be newly diagnosed, and more than 50,000 people will die from CRC \autocite{CancerStatistics}. Due to the severity of this issue to human health and society, biologists and mathematicians have turned their attention to understanding the mechanism of initiation and progression of this disease. 

The biological mechanism of colorectal cancer's initiation and propagation has been well studied. Several mutations must accumulate in a colonic crypt stem cell before giving rise to cancerous tumors. These include loss of chromosome maintenance, tumor suppressor loss of function (LOF), and proto-oncogene activation. 
Several pathways are known to contribute to colorectal cancer, but one dominant pathway occurs in 60\%-70\% of sporadic (non-inherited) tumors. The pathway begins with genomic instability, followed by two LOF mutations in each of two tumor-suppressing genes, APC and P$_{53}$ and one gain of function (GOF) mutation in both of the proto-oncogenes KRAS and PIK$_{3}$CA \autocite{PathwaysofCRC}.

% remove
%Observation has previously shown that APC mutations often occur upstream of KRAS mutations, and experiments have revealed a punitive effect of aspirin on cancerous cells (\autocite{GeneticAlterationsofMetastaticColorectalCancer},\autocite{}). Despite this, recent meta-analytical studies have revealed little effect of aspirin use on colorectal cancer initiation, and potentially negative effects of CRC mortality (McNeil et. al, 2018). The 
% remove 

As is typical of biological networks, this general pathway may involve several other events up and downstream of these mutations, and experiment may not reveal satisfying explanations for the observed trends.
Here, several questions emerge for mathematical biologists such as determining which processes must be included in a model of colorectal cancer initiation to align with observed trends, in what order do these events occur, and what preventative steps may be taken to slow cancer incidence.

This system has been studied mathematically by several groups. Methods to do so may be broadly grouped into deterministic or stochastic models, and spatial or compartmental models. \autocite{StochasticModel} Deterministic models have the advantage of being exactly reproducible, and are generally more simple in implementation, whereas stochastic models trade complexity to capture more variation in biological behavior. Spatial models of CRC capture a known biological element, which is the migration of cells outwards from stem cells. However, these models require many more parameters which are harder to estimate experimentally. Contrastingly, compartmental models ignore spatial correlations in favor of rigid transitions between states or compartments. Such models are useful in describing the dynamics of cell populations and their resistance or vulnerability to perturbation. 

The paper "Aspirin’s effect on kinetic parameters of cells contributes to its role in reducing incidence of advanced colorectal adenomas, shown by a multi scale computational study" by authors Wang, Boland, Goel, Wodarz, and Komarova, seeks to address open biological questions on the initiation of colorectal cancer. To do this, they build a compartmental deterministic model to investigate the dynamics of colonic crypts as they acquire mutations, compare their model predictions to epidemiological incidence, and test predictions regarding the mechanism of aspirin's effect on cancer incidence. This model consists of a system of ordinary differential equations (ODEs) to track unique crypt populations and an ODE describing the incidence probability of a fully mutated cancerous crypt. Lastly, they describe the effect of aspirin as a punishment on proliferation rate parameters \autocite{AspirinsEffect}.

To extend this work, we first built an understanding of the model's existing construction and behavior. Next, we linearized the ODE system about an equilibrium point by computing partial derivatives of each differential equation with respect to its dependent variables. Finally we conducted sensitivity analysis on the linearized system, and verified the results empirically. 

