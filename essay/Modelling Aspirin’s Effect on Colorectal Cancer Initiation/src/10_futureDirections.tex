\section{Future Directions}

There are multiple gaps in the current analysis that could be explored in future work. During linearization, we chose to only calculate around a single equilibrium point of zero. There are potentially other equilibrium points that could be worth solving for linearization. One approach could be to use Python’s SymPy toolbox. With this tool and with plugging in the chosen parameter set, we have managed to find two other nonnegative equilibrium points, aside from the zero vector point. The path has been laid for anyone wanting to fully explore the linearized models offered by this Jacobian linearization method. 

Another possibility is to use Lyapunov stability analysis to directly study the nonlinear ODE. Lyapunov stability is classically defined as a property of an equilibrium. When a system starts within a certain small distance from that equilibrium and it stays within that small distance forever, as opposed to converging or diverging, then that equilibrium is Lyapunov-stable \autocite{StabilityAndLyapunovStabilityOfDynamicalSystems}. Instead of solving around a specific equilibrium point, we can find the Lyapunov stability around the control black line, with respect to each colored line. If the control black line has the lowest Lyapunov stability with respect to any specific colored line, then it implies that the ODE system is the most vulnerable to perturbations in this crypt transition. One caveat to this method is that there are arguments against the use of Lyapunov analysis to study perturbations of populations \autocite{Justus_2008}.